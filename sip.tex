\documentclass[a4paper,10pt]{article}
\usepackage[utf8]{inputenc}
\usepackage[nottoc,numbib]{tocbibind} % makes the BibTeX references section appear in the table of contents
\usepackage[inline]{enumitem}
\usepackage{amssymb}
\usepackage{amsfonts}
\usepackage{amsmath}
\usepackage{amsthm}
\usepackage{color}
\usepackage{graphicx}
\usepackage{mathtools}
\usepackage{mathrsfs}
\usepackage{framed}
% eventually make left/right margins equal
\usepackage[top=0.6in,bottom=0.6in,left=0.1in,right=0.9in]{geometry}
\usepackage{dsfont}



\usepackage[textwidth=0.7in]{todonotes}

\usepackage[hidelinks]{hyperref} %this needs to be loaded last!

% for 3D pictures
\usepackage{tikz}
\usepackage{tikz-3dplot}
\usetikzlibrary{perspective}

%prevent math mode statements from being broken up across lines
\relpenalty=9999
\binoppenalty=9999

%claim environment
\newenvironment{claim}[1]{\par\noindent\underline{Claim:}\space#1}{}

%leftbar
\renewenvironment{leftbar}[1][\hsize]
{%
    \def\FrameCommand
    {%
        {\vrule width 3pt}%
        \hspace{0pt}%must no space.
        \fboxsep=\FrameSep\colorbox{white}%
    }%
    \MakeFramed{\hsize#1\advance\hsize-\width\FrameRestore}%
}
{\endMakeFramed}


%numbering and style of theorems
\theoremstyle{plain}
\newtheorem{Theorem}{Theorem}
\newtheorem{Proposition}[Theorem]{Proposition}
\newtheorem{Corollary}[Theorem]{Corollary}
\newtheorem{Lemma}[Theorem]{Lemma}
\newtheorem{Question}[Theorem]{Question}
\newtheorem{Conjecture}[Theorem]{Conjecture}
\newtheorem{Assumption}[Theorem]{Assumption}
\newtheorem{Algorithm}[Theorem]{Algorithm}

\theoremstyle{definition}
\newtheorem{Definition}[Theorem]{Definition}
\newtheorem{Property}[Theorem]{Property}
\newtheorem{Notation}[Theorem]{Notation}
\newtheorem{Condition}[Theorem]{Condition}
\newtheorem{Example}[Theorem]{Example}
\newtheorem{Exercise}[Theorem]{Exercise}
\newtheorem{Introduction}[Theorem]{Introduction}

\theoremstyle{remark}
\newtheorem{Remark}[Theorem]{Remark}
\newtheorem{case}{Case}[Theorem]

% make proofs use filled in black square instead of empty square
\renewcommand{\qedsymbol}{$\blacksquare$}
%\renewcommand\proof{\noindent\textit{\textbf{Proof. }}}
\newcommand{\modo}[3]{#1 \equiv #2 \pmod{#3}}
\newcommand{\nmodo}[3]{#1 \not\equiv #2 \pmod{#3}}
\newcommand{\R}{\mathbb{R}}
\newcommand{\Q}{\mathbb{Q}}
\newcommand{\N}{\mathbb{N}}
\newcommand{\Z}{\mathbb{Z}}
\newcommand{\Zpos}{\mathbb{Z}_{\geq 0}}
\renewcommand{\vec}[1]{\textbf{#1}}
\newcommand{\dist}{\text{dist}}
\newcommand{\F}{\mathbb{F}}
\newcommand{\C}{\mathbb{C}}
\newcommand{\lin}{\text{lin}}
\newcommand{\cl}{\text{cl}}
\newcommand\norm[1]{\left\lVert#1\right\rVert}
\newcommand{\ONE}{\mathds{1}}
\newcommand{\generatedby}[1]{\left\langle#1\right\rangle}
\newcommand{\id}{\text{id}}
\DeclareMathOperator{\sgn}{sgn}
\newcommand\abs[1]{\left|#1\right|}
\newcommand\Gl{\text{Gl}}

\title{Virus Research \\ \large SIP}
\author{Xavier Silva}

\begin{document}

\maketitle

\tableofcontents

% \newpage

\section{Abstract}

Icosahedral viruses have the symmetries of an icosahedron, which involves 2-fold, 3-fold, and 5-fold rotational symmetries.
We can approximate these virus capsids with finite sets of points (called a point array) which we realize in 6D (not just 3D) for the purpose of crystallography: our 6D point arrays naturally fit inside 6D icosahedral lattices. There are \(55\) standard point arrays (called \emph{one-base}) from which we build all the others.
We model virus maturation by 6D linear transformations (transitions) of point arrays that preserve some or all of icosahedral symmetry.
The symmetries we wish to preserve are either full icosahedral, $A_4$ ($3$-fold and $2$-fold), $D_{10}$ ($5$-fold and $2$-fold) or $D_6$ ($3$-fold and $2$-fold) symmetry.
We define full icosahedral symmetry as a group defined by \(\mathcal{I} := \generatedby{a, b | a^2 = b^3 = (ab)^5 = 1}\).
We then notice that \(A_4, D_{10}, \text{ and } D_6\) are maximal subgroups of \(\mathcal{I}\).
To find transitions that preserve one of these symmetry groups, we solve matrix equations of the form \(TB_0 = B_1\) for \(T\), where \(T\) is a \(6\times6\) matrix that depends on either 2, 4, 6, or 8 real variables, depending on which symmetry we are looking at and \(B_0\) and \(B_1\) are representations of the point arrays.
Note that actually finding transitions requires a large amount of computation.
From this we are able to reproduce previously discovered transitions for the Cowpea Chlorotic Mottle Virus that preserve \(D_6\) symmetry, and more importantly create a comprehesive list of what symmetries can be preserved between any possible combination of the \(55\) standard point arrays.

\section{Intro}
Icosahedral viruses have the symmetries of an icosahedron, which involves 2-fold, 3-fold, and 5-fold rotational symmetries.
We can approximate these virus capsids with finite sets of points (called a point array) which we realize in 6D (not just 3D) for the purpose of crystallography: our 6D point arrays naturally fit inside 6D icosahedral lattices. 
There are \(55\) standard point arrays (called \emph{one-base}) from which we build all the others.
We model virus maturation by 6D linear transformations (transitions) of point arrays that preserve some or all of icosahedral symmetry.


% Viruses have a point cloud, we can then lift into 6D dimensions in a non-trivial way.
% These point clouds will then fit into a \emph{lattice}.

Relevant to our work the shape of the icosahedron, a shape with 2-fold, 3-fold, and 5-fold symmetry.
\begin{Definition}[Icosahedral Group]
	The icosahedral group, which we denote by \(\mathcal{I}\), is the 60-element group given by \[\mathcal{I} := \generatedby{a, b | a^2 = b^3 = (ab)^5 = 1}.\]
\end{Definition}
\noindent We realize the icosahedral group as a \( 6 \times 6 \) matrix group.
\begin{Definition}[Generators of the Icosahedral Group]
	A 6D representation of the generators for \(\mathcal{I}\) are as follows:
	\[a = \begin{bmatrix}
		-1 & 0  & 0 & 0 & 0 & 0 \\
		0  & -1 & 0 & 0 & 0 & 0 \\
		0  & 0  & 0 & 0 & 1 & 0 \\
		0  & 0  & 0 & 0 & 0 & 1 \\
		0  & 0  & 1 & 0 & 0 & 0 \\
		0  & 0  & 0 & 1 & 0 & 0
	\end{bmatrix} \quad b = \begin{bmatrix}
		0 & -1 & 0  & 0 & 0 & 0 \\
		0 & 0  & -1 & 0 & 0 & 0 \\
		1 & 0  & 0  & 0 & 0 & 0 \\
		0 & 0  & 0  & 0 & 0 & 1 \\
		0 & 0  & 0  & 1 & 0 & 0 \\
		0 & 0  & 0  & 0 & 1 & 0
	\end{bmatrix}.\]
\end{Definition}

\section{The Problem}
Viruses can be approximated by a point cloud, which can be represented by a set of vectors in \(\R^6\).
\begin{Definition}[Point Array]
    Let \(\vec{t}, \vec{v}_1, \vec{v}_2, \dots, \vec{v}_n\) be vectors.
    Then the point array generated by these vectors is \[P = \mathcal{I}\vec{v}_1 \cup \mathcal{I}\vec{v}_2 \cup \dots \cup \mathcal{I}\vec{v}_n \cup (\mathcal{I}\vec{v}_1 + \mathcal{I}\vec{t}) \cup (\mathcal{I}\vec{v}_2 + \mathcal{I}\vec{t}) \cup \dots \cup (\mathcal{I}\vec{v}_n + \mathcal{I}\vec{t}).\]
\end{Definition}
\begin{Definition}[Admissible Basis]
    Given \(\vec{t}, \vec{v}_1, \vec{v}_2, \dots, \vec{v}_n\) be vectors that form a lifted viral configuration \(\Sigma\) with \(n \in \{1, \dots, 5\}\).
    Then any basis \(\{\vec{b}_\alpha\}_{\alpha = 1, \dots, 6}\) of \(\R^6\) such that
    \begin{enumerate}
        \item \(\{\vec{b}_\alpha\}\) is a basis for the minimal icosahedral lattice containing \(\Sigma\);
        \item every basis vector \(\vec{b}_\alpha\) belongs to one of \(\mathcal{I}\vec{t}, \mathcal{I}\vec{v}_1, \mathcal{I}\vec{v}_2, \dots, \text{or } \mathcal{I}\vec{v}_n\);
        \item each orbit contains at least one basis vector (that is, our basis represents each orbit);
    \end{enumerate}
    is called an \emph{admissible basis} for the lifted viral configuration.
\end{Definition}

\subsection{Abstract View of The Problem}
Forgetting about the virus stuff behind the scenes, lets state this problem in a more abstract way.
Let \(\vec{t}, \vec{v}_1, \vec{v}_2, \dots, \vec{v}_n\) be vectors that correspond to the start state.
Let \(\vec{t}', \vec{u}_1, \vec{u}_2, \dots, \vec{u}_m\) be vectors that correspond to the end state.
Let \(\beta_0\) and \(\beta_1\) be the set of possible admissible basis for the start state and end state respectively.
Let \(\mathcal{G}_0 = \mathcal{I}, A_4, D_{10}, \text{or } D_6\).
Then we wish to find \(T \in \mathcal{Z}(\mathcal{G}_0, \R)\) and \(B_0 \in \beta_0\) such that \[TB_0 \in \beta_1.\]
Furthermore, for each basis vector \(b_\alpha\) of \(B_0\), we require one of the following to hold:
\begin{itemize}
    \item \(b_\alpha \in \mathcal{I}\vec{t} \implies Tb_\alpha \in \mathcal{I}\vec{t}'\)
    \item \(b_\alpha \in \mathcal{I}\vec{v}_i \implies Tb_\alpha \in \mathcal{I}\vec{u}_j\) for some \(i, j\).
\end{itemize}

\section{Centralizers} \label{Centralizers}
A critical part of understanding the problem at hand is understanding centralizers of our relevant groups.
\begin{Definition}[General Linear Group]
    Let \(n \in \N\) and \(K = \Z, \Q, \text{or } \R\).
    Then \(\Gl(n, K)\) is the set of all \(n \times n\) invertible matrices.
    This forms a group.
\end{Definition}
\begin{Definition}[Centralizer]
    Let \(K = \Z, \Q, \text{or } \R\) and let \(\mathcal{G} \subseteq \Gl(6, K)\) be a matrix group.
    The centralizer of \(\mathcal{Z}(\mathcal{G}, K)\) of \(\mathcal{G}\) in \(\Gl(6, K)\) is
    \[\mathcal{Z}(\mathcal{G}, K) = \left\{N \in \Gl(6, K) | N^{-1}GN = G, \forall G \in \mathcal{G}\right\}.\]
    That is, the centralizer are the elements of \(\Gl(6, K)\) that commute with all elements of \(\mathcal{G}\).
\end{Definition}
The centralizers are precisely the Bain transitions we are looking for.
\todo[inline]{Describe why centralizers are the transitions we care about. They play nice with the symmetries because...}

As outlined in Indelcanto et al. \cite{indelicatoetal2012}, we can find the form of these centralizers by solving a set of linear equations.

\subsection{Tetrahedral Group \(A_4\)}
\[C = \begin{bmatrix}
    z  & -x & -y & -t & t  & -x \\
    t  & z  & t  & x  & x  & y  \\
    -y & -x & z  & t  & -t & -x \\
    x  & -t & -x & z  & y  & t  \\
    -x & -t & x  & y  & z  & t  \\
    t  & y  & t  & -x & -x & z 
\end{bmatrix}\]

\subsection{Dihedral Group \(D_{10}\)}
\[C = \begin{bmatrix}
    z & x & y & y & x & t \\
    x & z & x & y & y & t \\
    y & x & z & x & y & t \\
    y & y & x & z & x & t \\
    x & y & y & x & z & t \\
    u & u & u & u & u & w
\end{bmatrix}\]

\subsection{Dihedral Group \(D_{6}\)}
\[C = \begin{bmatrix}
    u  & w  & -w & x  & s  & s  \\
    -t & y  & v  & -v & z  & -t \\
    t  & v  & y  & v  & t  & -z \\
    z  & -v & v  & y  & -t & -t \\
    s  & x  & -w & w  & u  & s  \\
    s  & w  & -x & w  & s  & u 
\end{bmatrix}\]

\subsection{Icosahedral Group \(\mathcal{I}\)}
Since \(A_4\), \(D_{10}\), and \(D_6\) are all of the maximal subgroups of \(\mathcal{I}\), we can combine the forms of their centralizers to find that elements of \(\mathcal{Z}(\mathcal{I}, K)\) have the following form:
\[C = \begin{bmatrix}
    z  & x  & -x & -x & x  & x \\
    x  & z  & x  & -x & -x & x \\
    -x & x  & z  & x  & -x & x \\
    -x & -x & x  & z  & x  & x \\
    x  & -x & -x & x  & z  & x \\
    x  & x  & x  & x  & x  & z
\end{bmatrix}\]

\section{A General Procedure for Finding Bain Transitions} \label{A General Procedure for Finding Bain Transitions}
This section describes a generalization of the technique Indelcanto et al. \cite{indelicatoetal2012} used to find Bain transitions.
% begin essentially copying from Indelcanto
As Indelcanto et al. \cite{indelicatoetal2012} states, ``Bain transitions are special elements of the rational centralizer of the intermediate symmetry group."
And as shown in section \ref{Centralizers}, the centralizers of \(\mathcal{I}\) and its maximum subgroups in \(\Gl(6, \R)\) are \(6 \times 6\) matrices that depend on only a finite number of \(n\) variables.
% end essentially copying from Indelcanto

Write our viral transition as \(T = T(p_1, p_2, \dots, p_n)\), and \(\Sigma_0 = \Sigma(\vec{t}, \vec{v}_1, \dots, \vec{v}_k)\) and \(\Sigma_1 = \Sigma(\vec{t}', \vec{u}_1, \dots, \vec{u}_\ell)\), with \(k, \ell \in \{1, \dots, 5\}\).
Because our transition \(T\) is invertible and any \(6 \times 6\) matrix that represents an admissible bases is invertible, we are allowed to assume \(k \geq \ell\).

Now choose \(\overline{\vec{t}} \in \mathcal{I}\vec{t}\), \(\overline{\vec{v}_i} \in \mathcal{I}\vec{v}_i\), \(\overline{\vec{t}'} \in \mathcal{I}\vec{t}'\) and \(\overline{\vec{u}_j} \in \mathcal{I}\vec{u}_j\) for all \(i \in \{1, \dots, k\}\) and \(j \in \{1, \dots, \ell\}\).
Let \(\phi\colon \{1, \dots, k\} \to \{1, \dots, \ell\}\) be a surjective function.

Then we now require that
\begin{align}
    T(p_1, p_2, \dots, p_n)&\overline{\vec{t}} = \overline{\vec{t}'} \\
    T(p_1, p_2, \dots, p_n)&\overline{\vec{v}_i} = \overline{\vec{u}_{\phi(i)}} \quad\forall i \in \{1, \dots, k\}
\end{align}
This then gives us a system of equations for the \(p_1, p_2, \dots, p_n\).
If all of the following hold:
\begin{enumerate}
    \item solution for \(p_1, p_2, \dots, p_n\) exists;
    \item there exists an admissible basis \(B_0\) for \(\Sigma_0\) containing \(\overline{\vec{t}}, \overline{\vec{v}_1}, \dots, \overline{\vec{v}_k}\);
    \item \(TB_0\) is an admissible basis for \(\Sigma_1\) 
        containing \(\overline{\vec{t}'}, \overline{\vec{u}_1}, \dots, \overline{\vec{u}_{\ell}}\); % this part is given automatically by the equations above, but for now including anyways
\end{enumerate}
Then \(T\) is a viral transition between \(\Sigma_0\) and \(\Sigma_1\).

By repeating this procedure for all choices of \(\overline{\vec{t}}, \overline{\vec{v}_1}, \dots, \overline{\vec{v}_k}, \overline{\vec{t}'}, \overline{\vec{u}_1}, \dots, \overline{\vec{u}_\ell}\) from their respective orbits, we find all possible viral transitions.


\section{Techniques for Finding Transitions}
While the procedure outlined in section \ref{A General Procedure for Finding Bain Transitions} is valid and will find all transitions, using this procedure for actually computing transitions would take an unbearably long amount of time.
Suppose that we are working with the virus Turnip Crink Virus (TCV) for example.
The native state has one translation and three base vectors while the mature state has one translation and four base vectors.
Each of their icosahedral orbits has at least 12 elements.
So, when we go through all choices of vectors from their orbits, there are at least 5,159,780,352 systems of equations to solve for transitions for this example.
This is too much to compute.
Instead we have a change in approach, which involves working backwards.

\subsection{Entry Sampling}
As outlined \todo{Define what ``previously" stands for} {\color{red} previously}, we wish to find a matrix \(T \in \mathcal{Z}(\mathcal{G}, K)\) such that \(TB_0 = B_1\), where \(\mathcal{G} = \mathcal{I}, A_4, D_{10}, \text{or } D_6\).

We notice that when given a starting and ending configuration of a particular virus, we will know exactly what all admissible bases \(B_0\) and \(B_1\) look like \emph{and} we know what the form of \(T\) should be.
So if \(T\) is a Bain transition, we must have that the entries of \(T\) come from \(B_1B_0^{-1}\) for some admissible bases \(B_0\) and \(B_1\).
That is, if we define \(\text{Ent}\colon \Gl(6, K) \to \mathcal{P}(\R)\) to be the function the gives us the set of entries within a \(6 \times 6\) matrix, then we have that \[T \text{ is a transition } \implies \text{ there exists admissible bases \(B_0\) and \(B_1\) such that } \text{Ent}(T) \subseteq \text{Ent}(B_1B_0^{-1}).\]

Therefore if we compute all pairs of \(B_1B_0^{-1}\), we find that the entries of \(T\) only come from a finite list.
\footnote{In fact, based upon experimental data, checking a relatively small sample size (say 100,000) of \(B_0\) and \(B_1\) pairs gives us a mostly complete idea of the possible entries of \(T\) could be.}
\todo{Make this more clear / elaborate; give a proof?}
So there are only a finite number of Bain transitions that might possibly work.


\subsection{Partial Bain Transitions}
\begin{Definition}[Partial Bain Transition]
    Let \(\mathcal{G} = \mathcal{I}, A_4, D_{10}, \text{or } D_6\).
    A \emph{partial Bain transition} is a nonzero element of \(\mathcal{Z}(\mathcal{G}, K)\) where at least one row of the matrix is all zero.
\end{Definition}
Notice that there do not exist any partial Bain transitions for \(\mathcal{I}\) or \(A_4\), since forcing any row to be zero forces the matrix to be the zero matrix.

However, in the cases of \(D_{10}\) and \(D_6\), there do exist partial Bain transitions, and they turn out to be useful in finding Bain transitions.

For \(D_{10}\) we notice the following:
\[C = \begin{bmatrix}
    z & x & y & y & x & t \\
    x & z & x & y & y & t \\
    y & x & z & x & y & t \\
    y & y & x & z & x & t \\
    x & y & y & x & z & t \\
    u & u & u & u & u & w
\end{bmatrix} = \begin{bmatrix}
    z & x & y & y & x & t \\
    x & z & x & y & y & t \\
    y & x & z & x & y & t \\
    y & y & x & z & x & t \\
    x & y & y & x & z & t \\
    0 & 0 & 0 & 0 & 0 & 0
\end{bmatrix} + \begin{bmatrix}
    0 & 0 & 0 & 0 & 0 & 0 \\
    0 & 0 & 0 & 0 & 0 & 0 \\
    0 & 0 & 0 & 0 & 0 & 0 \\
    0 & 0 & 0 & 0 & 0 & 0 \\
    0 & 0 & 0 & 0 & 0 & 0 \\
    u & u & u & u & u & w
\end{bmatrix}.\]

And similarly for \(D_6\) we notice the following:
\[C = \begin{bmatrix}
    u  & w  & -w & x  & s  & s  \\
    -t & y  & v  & -v & z  & -t \\
    t  & v  & y  & v  & t  & -z \\
    z  & -v & v  & y  & -t & -t \\
    s  & x  & -w & w  & u  & s  \\
    s  & w  & -x & w  & s  & u 
\end{bmatrix} = \begin{bmatrix}
    u  & w  & -w & x  & s  & s  \\
    0 & 0 & 0 & 0 & 0 & 0 \\
    0 & 0 & 0 & 0 & 0 & 0 \\
    0 & 0 & 0 & 0 & 0 & 0 \\
    s  & x  & -w & w  & u  & s  \\
    s  & w  & -x & w  & s  & u 
\end{bmatrix} + \begin{bmatrix}
    0 & 0 & 0 & 0 & 0 & 0 \\
    -t & y  & v  & -v & z  & -t \\
    t  & v  & y  & v  & t  & -z \\
    z  & -v & v  & y  & -t & -t \\
    0 & 0 & 0 & 0 & 0 & 0 \\
    0 & 0 & 0 & 0 & 0 & 0
\end{bmatrix}.\]



\subsection{Using The Contrapositive}
We now have a list of Bain transitions that might work by entry sampling.
Now we need to find ways of telling when a transition matrix will not work besides trying to find \(B_0\) and \(B_1\) matrices that satisfy \(TB_0 = B_1\).

\begin{Lemma}
    Let \(P_0 = \mathcal{I}\vec{t} \cup \mathcal{I}\vec{v}_1 \cup \mathcal{I}\vec{v}_2 \cup \dots \cup \vec{v}_k\) and \(P_1 = \mathcal{I}\vec{t}' \cup \mathcal{I}\vec{u}_1 \cup \mathcal{I}\vec{u}_2 \cup \dots \cup \vec{v}_\ell\) and \(\mathcal{G} = \mathcal{I}, A_4, D_{10} \text{or } D_6\).
    If \(T\) is a transition that preservses \(\mathcal{G}\) symmetry, then there exists \(\vec{v} \in P_0\) such that, \[\norm{T\vec{v}} \leq \max_{\vec{u} \in P_1}\{\norm{\vec{u}}\}.\] 
\end{Lemma}

\todo[inline]{Add more of the conditions I use}

While these lemmas are fairly trivial, if we look at each of their contrapositives, then we get conditions that tell us when we do not have a transition.
This is useful because these conditions do not require us to attempt to make an admissible basis.

\subsection{Equation solving}
Another approach to solving this problem is to relax the requirements.
In reality, the viruses have multiple radial levels.
So there are multiple layers that each orbit is nested within.
Therefore to simplify the problem we can look at whether there exists a transition at the top layer only.
What this translates to in reality is that the outside of the virus has these symmetries preserves, but the insides might not preserve any form of symmetry.

In terms of our problem, this comes down to a 1-base problem.
So there is only the translation vector and one base vector.
This problem is much less computationally intensive, and it is now feasible to do what we wish we could do with the original problem, which is try all possible bases and solve the linear equation using the general form of the centralizer.

The type of equation we aim to solve using \(D_6\) and \(10 \to 27\) as an example:
\begingroup
\renewcommand*{\arraystretch}{1.5}
\[\displaystyle \left[\begin{matrix}u & w & - w & x & s & s\\- t & y & v & - v & z & - t\\t & v & y & v & t & - z\\z & - v & v & y & - t & t\\s & x & - w & w & u & s\\s & w & - x & w & s & u\end{matrix}\right]
\cdot
\displaystyle \left[\begin{matrix}\frac{1}{2} & \frac{3}{2}\\- \frac{1}{2} & - \frac{1}{2}\\\frac{1}{2} & \frac{1}{2}\\\frac{1}{2} & \frac{1}{2}\\- \frac{1}{2} & - \frac{1}{2}\\\frac{1}{2} & - \frac{1}{2}\end{matrix}\right]
=
\displaystyle \left[\begin{matrix}1 & 0\\0 & 0\\0 & 1\\0 & 1\\0 & 0\\0 & 1\end{matrix}\right]
\iff
\displaystyle \left[\begin{matrix}\frac{u}{2} - w + \frac{x}{2} & - s + \frac{3 u}{2} - w + \frac{x}{2}\\- t - \frac{y}{2} - \frac{z}{2} & - t - \frac{y}{2} - \frac{z}{2}\\\frac{y}{2} - \frac{z}{2} & t + \frac{y}{2} + \frac{z}{2}\\t + v + \frac{y}{2} + \frac{z}{2} & v + \frac{y}{2} + \frac{3 z}{2}\\s - \frac{u}{2} - \frac{x}{2} & s - \frac{u}{2} - \frac{x}{2}\\\frac{u}{2} - \frac{x}{2} & s - \frac{u}{2} - \frac{x}{2}\end{matrix}\right] = \left[\begin{matrix}1 & 0\\0 & 0\\0 & 1\\0 & 1\\0 & 0\\0 & 1\end{matrix}\right]\]
\endgroup

\section{Two Base Techniques}
First we understand that by two base, we mean a point array that comes from two individual base vectors and a shared transition vector.
That is, the point array looks like \[P = \mathcal{I}\vec{v} \cup \mathcal{I}\vec{u} \cup (\mathcal{I}\vec{v} + \mathcal{I}\vec{t}) \cup (\mathcal{I}\vec{u} + \mathcal{I}\vec{t}).\]
Suppose that we have vectors \(\vec{t}_0, \vec{v}_0, \vec{u}_0\) and vectors \(\vec{t}_1, \vec{v}_1, \vec{u}_1\) that generate point arrays \(P_0\) and \(P_1\) respectively.

\begin{Definition}[Bar Notation]
    \todo{Move this definition earlier}
    Let \(\vec{v}\) be a vector.
    Define \(\overline{\vec{v}}\) to be an element of the icosahedral orbit of \(\vec{v}\).
    That is, \(\overline{\vec{v}} \in \mathcal{I}\vec{v}\).
\end{Definition}
Then our goal for finding a two base transition requires us to find a transition \(T\) (which is one of general centralizer forms) and \(\overline{\vec{t}_0}, \overline{\vec{v}_0}, \overline{\vec{u}_0}, \overline{\vec{t}_1}, \overline{\vec{v}_1}, \overline{\vec{u}_1}\) such that
\begin{equation}
    \label{twobase-eq}
    T\begin{bmatrix}
        | & | & | \\
        \overline{\vec{t}_0} & \overline{\vec{v}_0} & \overline{\vec{u}_0} \\
        | & | & |
    \end{bmatrix} = \begin{bmatrix}
        | & | & | \\
        \overline{\vec{t}_1} & \overline{\vec{v}_1} & \overline{\vec{u}_1} \\
        | & | & |
    \end{bmatrix}
\end{equation}
\todo{Prove this} Note that if we can solve this, we can easily create \(B_0\) and \(B_1\) matrices of size \(6 \times 6\) since the remaining columns will be filled out by other members of the icosahedral orbit of either \(\vec{t}, \vec{v}, \vec{u}\) (with subscripts depending on if its in \(B_0\) or \(B_1\)).


Now finally here is how we find a two base transition for a given symmetry group \(\mathcal{G}\):
\begin{enumerate}
    \item First, find all transitions that solve the two separate one base cases individually.
        Define the set \(S_\vec{v}\) to be all matrices \(T_\vec{v}\) within \(Z(\mathcal{G}, K)\) such that \[T_\vec{v}\begin{bmatrix}
            | & | \\
            \overline{\vec{t}_0} & \overline{\vec{v}_0} \\
            | & |
        \end{bmatrix} = \begin{bmatrix}
            | & | \\
            \overline{\vec{t}_1} & \overline{\vec{v}_1} \\
            | & |
        \end{bmatrix}\] while running over all possibilities for \(\overline{\vec{t}_0}, \overline{\vec{v}_0}, \overline{\vec{t}_1}, \overline{\vec{v}_1}\).
        Since we know what the elements of \(Z(\mathcal{G}, K)\) look like (they are of a particular general form), we can somewhat easily find the set \(S_\vec{v}\).

        We define the \(S_\vec{u}\) in a similar way, those being all matrices \(T_\vec{u}\) within \(Z(\mathcal{G}, K)\) such that 
        \[T_\vec{u}\begin{bmatrix}
            | & | \\
            \overline{\vec{t}_0} & \overline{\vec{u}_0} \\
            | & |
        \end{bmatrix} = \begin{bmatrix}
            | & | \\
            \overline{\vec{t}_1} & \overline{\vec{u}_1} \\
            | & |
        \end{bmatrix}\] while running over all possibilities for \(\overline{\vec{t}_0}, \overline{\vec{u}_0}, \overline{\vec{t}_1}, \overline{\vec{u}_1}\).

    \item Now that we have these sets \(S_\vec{v}\) and \(S_\vec{u}\), we now look at \(S_\vec{v} \cap S_\vec{u}\).
    If this intersection is empty, then there cannot exist a transition matrix \(T\) such that \[T\begin{bmatrix}
    | & | & | \\
    \overline{\vec{t}_0} & \overline{\vec{v}_0} & \overline{\vec{u}_0} \\
    | & | & |
    \end{bmatrix} = \begin{bmatrix}
    | & | & | \\
    \overline{\vec{t}_1} & \overline{\vec{v}_1} & \overline{\vec{u}_1} \\
    | & | & |
    \end{bmatrix}.\]
    \item If this intersection is nonempty, then we finally need to check that we can actually find \(\overline{\vec{t}_0}, \overline{\vec{v}_0}, \overline{\vec{u}_0}, \overline{\vec{t}_1}, \overline{\vec{v}_1}, \overline{\vec{u}_1}\) that satisfy equation (\ref{twobase-eq}) for some \(T \in S_\vec{v} \cap S_\vec{u}\).
        While this step should be trivial, this step may fail if the only choices of vectors cause the \(6 \times 3\) matrices to not be full rank.
    \item If we can do all of these steps, we have found a two base transition between \(P_0\) and \(P_1\).
\end{enumerate}

\section{Final \texttt{Python} implementation}
This section describes the final method I implemented for finding virus translations.
The idea is to use a simple depth first search by building up the columns of the \(B_0\) and \(B_1\) matrices.
The program does the following:
\begin{enumerate}
    \item First we take in the point array numbers that represent our transition.
        Let our starting point array be represented by \((n_1, n_2, \dots, n_k)\) and our ending point array be represented by \((m_1, m_2, \dots, m_k)\).

    \item Next we get the generators for both of the point arrays. 
        We have it that \begin{gather*}
            (n_1, n_2, \dots, n_k) \mapsto (\vec{t}, \vec{v}_1, \vec{v}_2, \dots, \vec{v}_k) \\
            (m_1, m_2, \dots, m_k) \mapsto (\vec{t}', \vec{u}_1, \vec{u}_2, \dots, \vec{u}_k)
        \end{gather*}

    \item Then we create the \(\mathcal{I}\) orbits of the generators, giving us \begin{gather*}
        (\mathcal{I}\vec{t}, \mathcal{I}\vec{v}_1, \mathcal{I}\vec{v}_2, \dots, \mathcal{I}\vec{v}_k) \\ 
        (\mathcal{I}\vec{t}', \mathcal{I}\vec{u}_1, \mathcal{I}\vec{u}_2, \dots, \mathcal{I}\vec{u}_k)
        \end{gather*}

    \item Now that we have the \(\mathcal{I}\) orbits of the generators, we find all of possible vector pairs between the translation and base vectors.
        \todo{Not sure about the names of \(H_0\) and \(H_1\) for members of \(\mathcal{I}\)}
        That is, we find all \((h_0, h_1) \in \mathcal{I} \times \mathcal{I}\) such that \(Th_0\vec{t} = h_1\vec{t}'\) is solvable for \(T\).
        We do a similar process for the base vectors, we find all \((h_0, h_1) \in \mathcal{I} \times \mathcal{I}\) such that \(Th_0\vec{v}_i = h_1\vec{u}_i\) is solvable for \(T\).

    \item \todo[inline]{Define something about denoting subsets of \(\mathcal{I} \times \mathcal{I}\).}

    \item \todo[inline]{Talk about the depth first search of finding a transition \(T\) using the vector pairs (aka subsets of \(\mathcal{I} \times \mathcal{I}\)).}

    \item \todo[inline]{Talk about parallelization and memoization to optimize speed of program.}
\end{enumerate}

\section{Results}
After doing these computational methods, I found intriguing results, summarized by the following:
\begin{enumerate}
    \item These methods produce the same four transition matrices that \cite{indelicatoetal2012} find for the CCMV virus when preserving \(D_6\) symmetry. \label{results-reproduce-indelicato}
    
    \item Going beyond actual virus data, since we know that all of these icosahedral viruses will be built out of the standard 55 point arrays, we can create a comprehensive table of what is possible between any two pairs of the 55 point array.

    \item It is impossible for these icosahedral viruses to have a non-trivial transition that fully preserves icosahedral symmetry. We can only find transitions that preserve the maximal subgroups \(D_6, D_{10}, \text{ and } A_4\). \label{results-impossiblity-of-ico}

    \item \todo[inline]{Talk about 2+ base stuff.}
\end{enumerate}

\section{Acknowledgements}
\begin{itemize}
    \item Heyl Scholarship Research Fund
    \item Dr. Stephen Oloo and Dr. Dave Wilson
    \item Dr. Dave Wilson and Dr. Sandino Vargas-Perez for allowing me to use the Kalamazoo College supercomputer Jigwe.
\end{itemize}

\section{Poster Notes}
Let's use the \texttt{betterposter} template as seen in \href{https://www.youtube.com/watch?v=SYk29tnxASs}{this video (click for YouTube link)}.
A graphic to use would be something like
\begin{center}
\begin{tabular}{c|c|ccc}
    start & end & A4 & D10 & D6 \\ \hline
    1 & 2 & {\color{red}False} & {\color{red}False} & {\color{green}True} \\
    1 & 3 & {\color{red}False} & {\color{green}True} & {\color{green}True} \\
    1 & 4 & {\color{green}True} & {\color{red}False} & {\color{red}False} \\
    etc & etc & etc & etc & etc
\end{tabular}
\end{center}
with the True and False being just colored boxes.

Emphasize that while I am just solving linear equations, I am solving millions of these linear equations (compute a number?)

\medskip
\bibliographystyle{unsrt}%Used BibTeX style is unsrt
\bibliography{references}

\end{document}
