\documentclass[a4paper,10pt]{article}
\usepackage[utf8]{inputenc}
\usepackage[nottoc,numbib]{tocbibind} % makes the BibTeX references section appear in the table of contents
\usepackage[inline]{enumitem}
\usepackage{amssymb}
\usepackage{amsfonts}
\usepackage{amsmath}
\usepackage{amsthm}
\usepackage{color}
\usepackage{graphicx}
\usepackage{mathtools}
\usepackage{mathrsfs}
\usepackage{framed}
% eventually make left/right margins equal
\usepackage[top=0.6in,bottom=0.6in,left=0.1in,right=0.9in]{geometry}
\usepackage{dsfont}



\usepackage[textwidth=0.7in]{todonotes}

\usepackage[hidelinks]{hyperref} %this needs to be loaded last!

% for 3D pictures
\usepackage{tikz}
\usepackage{tikz-3dplot}
\usetikzlibrary{perspective}

%prevent math mode statements from being broken up across lines
\relpenalty=9999
\binoppenalty=9999

%claim environment
\newenvironment{claim}[1]{\par\noindent\underline{Claim:}\space#1}{}

%leftbar
\renewenvironment{leftbar}[1][\hsize]
{%
    \def\FrameCommand
    {%
        {\vrule width 3pt}%
        \hspace{0pt}%must no space.
        \fboxsep=\FrameSep\colorbox{white}%
    }%
    \MakeFramed{\hsize#1\advance\hsize-\width\FrameRestore}%
}
{\endMakeFramed}


%numbering and style of theorems
\theoremstyle{plain}
\newtheorem{Theorem}{Theorem}
\newtheorem{Proposition}[Theorem]{Proposition}
\newtheorem{Corollary}[Theorem]{Corollary}
\newtheorem{Lemma}[Theorem]{Lemma}
\newtheorem{Question}[Theorem]{Question}
\newtheorem{Conjecture}[Theorem]{Conjecture}
\newtheorem{Assumption}[Theorem]{Assumption}
\newtheorem{Algorithm}[Theorem]{Algorithm}

\theoremstyle{definition}
\newtheorem{Definition}[Theorem]{Definition}
\newtheorem{Property}[Theorem]{Property}
\newtheorem{Notation}[Theorem]{Notation}
\newtheorem{Condition}[Theorem]{Condition}
\newtheorem{Example}[Theorem]{Example}
\newtheorem{Exercise}[Theorem]{Exercise}
\newtheorem{Introduction}[Theorem]{Introduction}

\theoremstyle{remark}
\newtheorem{Remark}[Theorem]{Remark}
\newtheorem{case}{Case}[Theorem]

% make proofs use filled in black square instead of empty square
\renewcommand{\qedsymbol}{$\blacksquare$}
%\renewcommand\proof{\noindent\textit{\textbf{Proof. }}}
\newcommand{\modo}[3]{#1 \equiv #2 \pmod{#3}}
\newcommand{\nmodo}[3]{#1 \not\equiv #2 \pmod{#3}}
\newcommand{\R}{\mathbb{R}}
\newcommand{\Q}{\mathbb{Q}}
\newcommand{\N}{\mathbb{N}}
\newcommand{\Z}{\mathbb{Z}}
\newcommand{\Zpos}{\mathbb{Z}_{\geq 0}}
\renewcommand{\vec}[1]{\textbf{#1}}
\newcommand{\dist}{\text{dist}}
\newcommand{\F}{\mathbb{F}}
\newcommand{\C}{\mathbb{C}}
\newcommand{\lin}{\text{lin}}
\newcommand{\cl}{\text{cl}}
\newcommand\norm[1]{\left\lVert#1\right\rVert}
\newcommand{\ONE}{\mathds{1}}
\newcommand{\generatedby}[1]{\left\langle#1\right\rangle}
\newcommand{\id}{\text{id}}
\DeclareMathOperator{\sgn}{sgn}
\newcommand\abs[1]{\left|#1\right|}
\newcommand\Gl{\text{Gl}}

\title{Virus Research \\ \large SIP}
\author{Xavier Silva}

\begin{document}

\maketitle

\tableofcontents

% \newpage

\section{Abstract}

Icosahedral viruses have the symmetries of an icosahedron, which involves 2-fold, 3-fold, and 5-fold rotational symmetries.
We can approximate these virus capsids with finite sets of points (called a point array) which we realize in 6D (not just 3D) for the purpose of crystallography: our 6D point arrays naturally fit inside 6D icosahedral lattices. There are \(55\) standard point arrays (called \emph{one-base}) from which we build all the others.
We model virus maturation by 6D linear transformations (transitions) of point arrays that preserve some or all of icosahedral symmetry.
The symmetries we wish to preserve are either full icosahedral, $A_4$ ($3$-fold and $2$-fold), $D_{10}$ ($5$-fold and $2$-fold) or $D_6$ ($3$-fold and $2$-fold) symmetry.
We define full icosahedral symmetry as a group defined by \(\mathcal{I} := \generatedby{a, b | a^2 = b^3 = (ab)^5 = 1}\).
We then notice that \(A_4, D_{10}, \text{ and } D_6\) are maximal subgroups of \(\mathcal{I}\).
To find transitions that preserve one of these symmetry groups, we solve matrix equations of the form \(TB_0 = B_1\) for \(T\), where \(T\) is a \(6\times6\) matrix that depends on either 2, 4, 6, or 8 real variables, depending on which symmetry we are looking at and \(B_0\) and \(B_1\) are representations of the point arrays.
Note that actually finding transitions requires a large amount of computation.
From this we are able to reproduce previously discovered transitions for the Cowpea Chlorotic Mottle Virus that preserve \(D_6\) symmetry, and more importantly create a comprehesive list of what symmetries can be preserved between any possible combination of the \(55\) standard point arrays.

\section{Intro}
Icosahedral viruses have the symmetries of an icosahedron, which involves 2-fold, 3-fold, and 5-fold rotational symmetries.
We can approximate these virus capsids with finite sets of points (called a point array) which we realize in 6D (not just 3D) for the purpose of crystallography: our 6D point arrays naturally fit inside 6D icosahedral lattices. 
There are \(55\) standard point arrays (called \emph{one-base}) from which we build all the others.
We model virus maturation by 6D linear transformations (transitions) of point arrays that preserve some or all of icosahedral symmetry.


% Viruses have a point cloud, we can then lift into 6D dimensions in a non-trivial way.
% These point clouds will then fit into a \emph{lattice}.

Relevant to our work the shape of the icosahedron, a shape with 2-fold, 3-fold, and 5-fold symmetry.
\begin{Definition}[Icosahedral Group]
	The icosahedral group, which we denote by \(\mathcal{I}\), is the 60-element group given by \[\mathcal{I} := \generatedby{a, b | a^2 = b^3 = (ab)^5 = 1}.\]
\end{Definition}
\noindent We realize the icosahedral group as a \( 6 \times 6 \) matrix group.
\begin{Definition}[Generators of the Icosahedral Group]
	A 6D representation of the generators for \(\mathcal{I}\) are as follows:
	\[a = \begin{bmatrix}
		-1 & 0  & 0 & 0 & 0 & 0 \\
		0  & -1 & 0 & 0 & 0 & 0 \\
		0  & 0  & 0 & 0 & 1 & 0 \\
		0  & 0  & 0 & 0 & 0 & 1 \\
		0  & 0  & 1 & 0 & 0 & 0 \\
		0  & 0  & 0 & 1 & 0 & 0
	\end{bmatrix} \quad b = \begin{bmatrix}
		0 & -1 & 0  & 0 & 0 & 0 \\
		0 & 0  & -1 & 0 & 0 & 0 \\
		1 & 0  & 0  & 0 & 0 & 0 \\
		0 & 0  & 0  & 0 & 0 & 1 \\
		0 & 0  & 0  & 1 & 0 & 0 \\
		0 & 0  & 0  & 0 & 1 & 0
	\end{bmatrix}.\]
\end{Definition}

\section{Point Arrays and Lattices}

\subsection{Constructing Point Arrays}

\section{Transitions Between Point Arrays}
\subsection{Viral Maturation}
\todo[inline]{Give a less mathy description of the virus transition problem \\ describe goals about preserving symmetry}

\section{Mathematical View of the Problem}

\subsection{Mathematically Describing Transitions}

\subsection{Transitions Preserving Symmetry}

% order of these two sections?
\subsection{How To Find Transitions}
\subsection{CCMV \( D_6 \) Transition Example}


\section{Computational Techniques}
\todo[inline]{
	Describe the nature of this problem computationally \\ 
	- Embarassingly parallel \\
	- How to store this data \\
	- More...?
}

\todo[inline]{Talk about how we can build up \( B0 \) and \( B1 \) matrices one vector at a time}

\subsection{C++ Program}
\subsubsection{Entry Sampling}
\subsubsection{Partial Transitions}
\subsubsection{Contrapositive}
\todo[inline]{- Norm checking\\
- Checking mapping into the ending point array
}

\subsubsection{Parallelization}

\subsection{Python Program}
\subsubsection{Pair Checking}
\todo[inline]{this will involve the subsets of \( \mathcal{I} \times \mathcal{I} \) idea}
\subsubsection{Depth first search}
\todo[inline]{this is the current version running on Jigwe}
\subsubsection{Parallelization}

\section{Results}

\section{Acknowledgements}
\begin{itemize}
	\item Heyl Scholarship Research Fund
	\item Dr. Stephen Oloo and Dr. Dave Wilson
	\item Dr. Dave Wilson and Dr. Sandino Vargas-Perez for allowing me to use the Kalamazoo College supercomputer Jigwe.
\end{itemize}


\medskip
\bibliographystyle{unsrt}%Used BibTeX style is unsrt
\bibliography{references}

\end{document}
